\documentclass{article}
\usepackage{amssymb}
\usepackage[utf8]{inputenc}
\usepackage[total={17cm,25cm}, top=3cm, left=2cm, includefoot]{geometry}



\begin{document}
\pagenumbering{gobble}
\title{VisInGraph}
\author{Petr Chmel}
\date{}

\maketitle
\section{Introduction}
VisInGraph is a C++ program for visualising intersection graphs.
Currently, it is able to visualise so-called $\{\llcorner\}$-graphs using two different methods.

\section{Usage}
The program is distributed in a ZIP file for Windows OS.
It is sufficient to extract the ZIP file contents to a folder of user's choosing and then run the main program (VisInGraph.exe).

There are white areas: the left one is used for inputting the graph and the right one displays the intersection graph representation, if it exists.
Under these areas, there is a select box, which shows all possible graph representations, and the ``convert'' button, which start the conversion process.

Should the calculation take a longer amount of time, a progress bar is displayed to show progress.

\section{The program architecture}
The program has two main parts: the GUI and the graph generation.
There are also auxiliary classes, which are simple implementations of standard mathematical structures (namely an undirected graph and a poset).

The program also uses Qt library for its GUI.

\subsection{GUI}
The Graphical User Interface is defined in classes \texttt{MainWindow} and \texttt{MainWidget}.
The former class encapsulates the latter, which is responsible for user interaction.


\subsection{Graph generation}
The graph generation part defines two abstract classes \texttt{GraphRepresentation} and \texttt{GraphRepresentationManager}.

A class, which inherits from \texttt{GraphRepresentation} should correspond to a type of intersection graph.
The only requirement is, that if the representation exists, it must be able to be drawn (i.e. it must implement the method \texttt{draw(QGraphicsView\&)}.
Currently, VisInGraph only has one child class: \texttt{LGraphRepresentation}.

The main calculation part will generally take place in the class inheriting from \texttt{GraphRepresentationManager}.
Such class has to implement methods \texttt{generateFromGraph(Graph\&)}, \texttt{draw(QGraphicsView\&)} and the slot \texttt{stopCalculation()}.
The slot is used to abort calculations, so after the slot receives a signal, it should stop the calculation in a reasonable amount of time.
Currently, VisInGraph has two child classes: \texttt{LGraphRepresentationManager} and \texttt{LGraphExtRepresentationManager}.


\subsection{Auxiliary classes}
There are two main auxiliary classes: \texttt{Graph} and \texttt{Poset}.
The former implements a graph with the possibility of adding or removing a vertex or an edge with methods \texttt{addVertex()}, \texttt{removeVertex(v)}, \texttt{addEdge(u,v)}, \texttt{removeEdge(u,v)}.
The latter simply implements a poset - the basic methods are \texttt{addInequality(a,b)}, which adds the inequality $a\leq b$, \texttt{lte(a,b)}, which returns true if and only if $a\leq b$, and \texttt{comparable(a,b)}, which return true if and only if $a\leq b\lor a\geq b$.


\subsection{$\{\llcorner\}$-graph algorithms}
First algorithm builds the $\{\llcorner\}$-graph by trying all possible permutations of vertices on the x-axis and y-axis and for each pair of permutations, tries to construct the $\{\llcorner\}$-graph.

The second algorithm doesn't check all possible pairs of permutations, but for each x-axis permutation, it constructs a poset, from which all of its linear extensions are generated using an algorithm by Ono and Nakano \cite{posetAlgo}.
These linear extension are then used as the second part of the pair.


\section{Extending the program for other graph classes}
For any other graph class, it is necessary to create a new child class of \texttt{GraphRepresentation} and \texttt{GraphRepresentationManager}.
Then, the implemented generation should be added to the GUI.
First, add another value to the enum class \texttt{Representation} (found in \texttt{mainwidget.h}) and then add the choice to \texttt{MainWidget::initChoices()} and \texttt{MainWidget::convert()}, following suit of the previous classes.

\bibliographystyle{plain}
\bibliography{docs}


\end{document}